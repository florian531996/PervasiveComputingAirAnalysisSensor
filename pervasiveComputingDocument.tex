
%% bare_conf.tex
%% V1.3
%% 2007/01/11
%% by Michael Shell
%% See:
%% http://www.michaelshell.org/
%% for current contact information.
%%
%% This is a skeleton file demonstrating the use of IEEEtran.cls
%% (requires IEEEtran.cls version 1.7 or later) with an IEEE conference paper.
%%
%% Support sites:
%% http://www.michaelshell.org/tex/ieeetran/
%% http://www.ctan.org/tex-archive/macros/latex/contrib/IEEEtran/
%% and
%% http://www.ieee.org/

%%*************************************************************************
%% Legal Notice:
%% This code is offered as-is without any warranty either expressed or
%% implied; without even the implied warranty of MERCHANTABILITY or
%% FITNESS FOR A PARTICULAR PURPOSE! 
%% User assumes all risk.
%% In no event shall IEEE or any contributor to this code be liable for
%% any damages or losses, including, but not limited to, incidental,
%% consequential, or any other damages, resulting from the use or misuse
%% of any information contained here.
%%
%% All comments are the opinions of their respective authors and are not
%% necessarily endorsed by the IEEE.
%%
%% This work is distributed under the LaTeX Project Public License (LPPL)
%% ( http://www.latex-project.org/ ) version 1.3, and may be freely used,
%% distributed and modified. A copy of the LPPL, version 1.3, is included
%% in the base LaTeX documentation of all distributions of LaTeX released
%% 2003/12/01 or later.
%% Retain all contribution notices and credits.
%% ** Modified files should be clearly indicated as such, including  **
%% ** renaming them and changing author support contact information. **
%%
%% File list of work: IEEEtran.cls, IEEEtran_HOWTO.pdf, bare_adv.tex,
%%                    bare_conf.tex, bare_jrnl.tex, bare_jrnl_compsoc.tex
%%*************************************************************************

% *** Authors should verify (and, if needed, correct) their LaTeX system  ***
% *** with the testflow diagnostic prior to trusting their LaTeX platform ***
% *** with production work. IEEE's font choices can trigger bugs that do  ***
% *** not appear when using other class files.                            ***
% The testflow support page is at:
% http://www.michaelshell.org/tex/testflow/



% Note that the a4paper option is mainly intended so that authors in
% countries using A4 can easily print to A4 and see how their papers will
% look in print - the typesetting of the document will not typically be
% affected with changes in paper size (but the bottom and side margins will).
% Use the testflow package mentioned above to verify correct handling of
% both paper sizes by the user's LaTeX system.
%
% Also note that the "draftcls" or "draftclsnofoot", not "draft", option
% should be used if it is desired that the figures are to be displayed in
% draft mode.
%
\documentclass[conference]{IEEEtran}
% Add the compsoc option for Computer Society conferences.
%
% If IEEEtran.cls has not been installed into the LaTeX system files,
% manually specify the path to it like:
% \documentclass[conference]{../sty/IEEEtran}





% Some very useful LaTeX packages include:
% (uncomment the ones you want to load)


% *** MISC UTILITY PACKAGES ***
%
%\usepackage{ifpdf}
% Heiko Oberdiek's ifpdf.sty is very useful if you need conditional
% compilation based on whether the output is pdf or dvi.
% usage:
% \ifpdf
%   % pdf code
% \else
%   % dvi code
% \fi
% The latest version of ifpdf.sty can be obtained from:
% http://www.ctan.org/tex-archive/macros/latex/contrib/oberdiek/
% Also, note that IEEEtran.cls V1.7 and later provides a builtin
% \ifCLASSINFOpdf conditional that works the same way.
% When switching from latex to pdflatex and vice-versa, the compiler may
% have to be run twice to clear warning/error messages.






% *** CITATION PACKAGES ***
%
%\usepackage{cite}
% cite.sty was written by Donald Arseneau
% V1.6 and later of IEEEtran pre-defines the format of the cite.sty package
% \cite{} output to follow that of IEEE. Loading the cite package will
% result in citation numbers being automatically sorted and properly
% "compressed/ranged". e.g., [1], [9], [2], [7], [5], [6] without using
% cite.sty will become [1], [2], [5]--[7], [9] using cite.sty. cite.sty's
% \cite will automatically add leading space, if needed. Use cite.sty's
% noadjust option (cite.sty V3.8 and later) if you want to turn this off.
% cite.sty is already installed on most LaTeX systems. Be sure and use
% version 4.0 (2003-05-27) and later if using hyperref.sty. cite.sty does
% not currently provide for hyperlinked citations.
% The latest version can be obtained at:
% http://www.ctan.org/tex-archive/macros/latex/contrib/cite/
% The documentation is contained in the cite.sty file itself.






% *** GRAPHICS RELATED PACKAGES ***
%
\ifCLASSINFOpdf
  % \usepackage[pdftex]{graphicx}
  % declare the path(s) where your graphic files are
  % \graphicspath{{../pdf/}{../jpeg/}}
  % and their extensions so you won't have to specify these with
  % every instance of \includegraphics
  % \DeclareGraphicsExtensions{.pdf,.jpeg,.png}
\else
  % or other class option (dvipsone, dvipdf, if not using dvips). graphicx
  % will default to the driver specified in the system graphics.cfg if no
  % driver is specified.
  % \usepackage[dvips]{graphicx}
  % declare the path(s) where your graphic files are
  % \graphicspath{{../eps/}}
  % and their extensions so you won't have to specify these with
  % every instance of \includegraphics
  % \DeclareGraphicsExtensions{.eps}
\fi
% graphicx was written by David Carlisle and Sebastian Rahtz. It is
% required if you want graphics, photos, etc. graphicx.sty is already
% installed on most LaTeX systems. The latest version and documentation can
% be obtained at: 
% http://www.ctan.org/tex-archive/macros/latex/required/graphics/
% Another good source of documentation is "Using Imported Graphics in
% LaTeX2e" by Keith Reckdahl which can be found as epslatex.ps or
% epslatex.pdf at: http://www.ctan.org/tex-archive/info/
%
% latex, and pdflatex in dvi mode, support graphics in encapsulated
% postscript (.eps) format. pdflatex in pdf mode supports graphics
% in .pdf, .jpeg, .png and .mps (metapost) formats. Users should ensure
% that all non-photo figures use a vector format (.eps, .pdf, .mps) and
% not a bitmapped formats (.jpeg, .png). IEEE frowns on bitmapped formats
% which can result in "jaggedy"/blurry rendering of lines and letters as
% well as large increases in file sizes.
%
% You can find documentation about the pdfTeX application at:
% http://www.tug.org/applications/pdftex

\usepackage{graphicx}
\usepackage{tabularx}
\usepackage[strings]{underscore}
\usepackage{fixltx2e}

% *** MATH PACKAGES ***
%
%\usepackage[cmex10]{amsmath}
% A popular package from the American Mathematical Society that provides
% many useful and powerful commands for dealing with mathematics. If using
% it, be sure to load this package with the cmex10 option to ensure that
% only type 1 fonts will utilized at all point sizes. Without this option,
% it is possible that some math symbols, particularly those within
% footnotes, will be rendered in bitmap form which will result in a
% document that can not be IEEE Xplore compliant!
%
% Also, note that the amsmath package sets \interdisplaylinepenalty to 10000
% thus preventing page breaks from occurring within multiline equations. Use:
%\interdisplaylinepenalty=2500
% after loading amsmath to restore such page breaks as IEEEtran.cls normally
% does. amsmath.sty is already installed on most LaTeX systems. The latest
% version and documentation can be obtained at:
% http://www.ctan.org/tex-archive/macros/latex/required/amslatex/math/





% *** SPECIALIZED LIST PACKAGES ***
%
%\usepackage{algorithmic}
% algorithmic.sty was written by Peter Williams and Rogerio Brito.
% This package provides an algorithmic environment fo describing algorithms.
% You can use the algorithmic environment in-text or within a figure
% environment to provide for a floating algorithm. Do NOT use the algorithm
% floating environment provided by algorithm.sty (by the same authors) or
% algorithm2e.sty (by Christophe Fiorio) as IEEE does not use dedicated
% algorithm float types and packages that provide these will not provide
% correct IEEE style captions. The latest version and documentation of
% algorithmic.sty can be obtained at:
% http://www.ctan.org/tex-archive/macros/latex/contrib/algorithms/
% There is also a support site at:
% http://algorithms.berlios.de/index.html
% Also of interest may be the (relatively newer and more customizable)
% algorithmicx.sty package by Szasz Janos:
% http://www.ctan.org/tex-archive/macros/latex/contrib/algorithmicx/




% *** ALIGNMENT PACKAGES ***
%
%\usepackage{array}
% Frank Mittelbach's and David Carlisle's array.sty patches and improves
% the standard LaTeX2e array and tabular environments to provide better
% appearance and additional user controls. As the default LaTeX2e table
% generation code is lacking to the point of almost being broken with
% respect to the quality of the end results, all users are strongly
% advised to use an enhanced (at the very least that provided by array.sty)
% set of table tools. array.sty is already installed on most systems. The
% latest version and documentation can be obtained at:
% http://www.ctan.org/tex-archive/macros/latex/required/tools/


%\usepackage{mdwmath}
%\usepackage{mdwtab}
% Also highly recommended is Mark Wooding's extremely powerful MDW tools,
% especially mdwmath.sty and mdwtab.sty which are used to format equations
% and tables, respectively. The MDWtools set is already installed on most
% LaTeX systems. The lastest version and documentation is available at:
% http://www.ctan.org/tex-archive/macros/latex/contrib/mdwtools/


% IEEEtran contains the IEEEeqnarray family of commands that can be used to
% generate multiline equations as well as matrices, tables, etc., of high
% quality.


%\usepackage{eqparbox}
% Also of notable interest is Scott Pakin's eqparbox package for creating
% (automatically sized) equal width boxes - aka "natural width parboxes".
% Available at:
% http://www.ctan.org/tex-archive/macros/latex/contrib/eqparbox/





% *** SUBFIGURE PACKAGES ***
%\usepackage[tight,footnotesize]{subfigure}
% subfigure.sty was written by Steven Douglas Cochran. This package makes it
% easy to put subfigures in your figures. e.g., "Figure 1a and 1b". For IEEE
% work, it is a good idea to load it with the tight package option to reduce
% the amount of white space around the subfigures. subfigure.sty is already
% installed on most LaTeX systems. The latest version and documentation can
% be obtained at:
% http://www.ctan.org/tex-archive/obsolete/macros/latex/contrib/subfigure/
% subfigure.sty has been superceeded by subfig.sty.



%\usepackage[caption=false]{caption}
%\usepackage[font=footnotesize]{subfig}
% subfig.sty, also written by Steven Douglas Cochran, is the modern
% replacement for subfigure.sty. However, subfig.sty requires and
% automatically loads Axel Sommerfeldt's caption.sty which will override
% IEEEtran.cls handling of captions and this will result in nonIEEE style
% figure/table captions. To prevent this problem, be sure and preload
% caption.sty with its "caption=false" package option. This is will preserve
% IEEEtran.cls handing of captions. Version 1.3 (2005/06/28) and later 
% (recommended due to many improvements over 1.2) of subfig.sty supports
% the caption=false option directly:
%\usepackage[caption=false,font=footnotesize]{subfig}
%
% The latest version and documentation can be obtained at:
% http://www.ctan.org/tex-archive/macros/latex/contrib/subfig/
% The latest version and documentation of caption.sty can be obtained at:
% http://www.ctan.org/tex-archive/macros/latex/contrib/caption/




% *** FLOAT PACKAGES ***
%
%\usepackage{fixltx2e}
% fixltx2e, the successor to the earlier fix2col.sty, was written by
% Frank Mittelbach and David Carlisle. This package corrects a few problems
% in the LaTeX2e kernel, the most notable of which is that in current
% LaTeX2e releases, the ordering of single and double column floats is not
% guaranteed to be preserved. Thus, an unpatched LaTeX2e can allow a
% single column figure to be placed prior to an earlier double column
% figure. The latest version and documentation can be found at:
% http://www.ctan.org/tex-archive/macros/latex/base/



%\usepackage{stfloats}
% stfloats.sty was written by Sigitas Tolusis. This package gives LaTeX2e
% the ability to do double column floats at the bottom of the page as well
% as the top. (e.g., "\begin{figure*}[!b]" is not normally possible in
% LaTeX2e). It also provides a command:
%\fnbelowfloat
% to enable the placement of footnotes below bottom floats (the standard
% LaTeX2e kernel puts them above bottom floats). This is an invasive package
% which rewrites many portions of the LaTeX2e float routines. It may not work
% with other packages that modify the LaTeX2e float routines. The latest
% version and documentation can be obtained at:
% http://www.ctan.org/tex-archive/macros/latex/contrib/sttools/
% Documentation is contained in the stfloats.sty comments as well as in the
% presfull.pdf file. Do not use the stfloats baselinefloat ability as IEEE
% does not allow \baselineskip to stretch. Authors submitting work to the
% IEEE should note that IEEE rarely uses double column equations and
% that authors should try to avoid such use. Do not be tempted to use the
% cuted.sty or midfloat.sty packages (also by Sigitas Tolusis) as IEEE does
% not format its papers in such ways.





% *** PDF, URL AND HYPERLINK PACKAGES ***
%
%\usepackage{url}
% url.sty was written by Donald Arseneau. It provides better support for
% handling and breaking URLs. url.sty is already installed on most LaTeX
% systems. The latest version can be obtained at:
% http://www.ctan.org/tex-archive/macros/latex/contrib/misc/
% Read the url.sty source comments for usage information. Basically,
% \url{my_url_here}.





% *** Do not adjust lengths that control margins, column widths, etc. ***
% *** Do not use packages that alter fonts (such as pslatex).         ***
% There should be no need to do such things with IEEEtran.cls V1.6 and later.
% (Unless specifically asked to do so by the journal or conference you plan
% to submit to, of course. )


% correct bad hyphenation here
\hyphenation{op-tical net-works semi-conduc-tor}


\begin{document}
%
% paper title
% can use linebreaks \\ within to get better formatting as desired
\title{Air quality sensor}


% author names and affiliations
% use a multiple column layout for up to three different
% affiliations
\author{\IEEEauthorblockN{Henrik Lechte, Florian Finkel, Julia Grabinski, Cara Damm}
\IEEEauthorblockA{University of Mannheim\\
Pervasive Computing: Smart City Air Quality Sensors - FSS 2019\\
Master of Business Informatics\\ 
Email: \{first name\}.\{last name\}@mail.uni-mannheim.de}
}

% conference papers do not typically use \thanks and this command
% is locked out in conference mode. If really needed, such as for
% the acknowledgment of grants, issue a \IEEEoverridecommandlockouts
% after \documentclass

% for over three affiliations, or if they all won't fit within the width
% of the page, use this alternative format:
% 
%\author{\IEEEauthorblockN{Michael Shell\IEEEauthorrefmark{1},
%Homer Simpson\IEEEauthorrefmark{2},
%James Kirk\IEEEauthorrefmark{3}, 
%Montgomery Scott\IEEEauthorrefmark{3} and
%Eldon Tyrell\IEEEauthorrefmark{4}}
%\IEEEauthorblockA{\IEEEauthorrefmark{1}School of Electrical and Computer Engineering\\
%Georgia Institute of Technology,
%Atlanta, Georgia 30332--0250\\ Email: see http://www.michaelshell.org/contact.html}
%\IEEEauthorblockA{\IEEEauthorrefmark{2}Twentieth Century Fox, Springfield, USA\\
%Email: homer@thesimpsons.com}
%\IEEEauthorblockA{\IEEEauthorrefmark{3}Starfleet Academy, San Francisco, California 96678-2391\\
%Telephone: (800) 555--1212, Fax: (888) 555--1212}
%\IEEEauthorblockA{\IEEEauthorrefmark{4}Tyrell Inc., 123 Replicant Street, Los Angeles, California 90210--4321}}




% use for special paper notices
%\IEEEspecialpapernotice{(Invited Paper)}




% make the title area
\maketitle
\thispagestyle{plain}
\pagestyle{plain}


\begin{abstract}
%\boldmath
\end{abstract}
% IEEEtran.cls defaults to using nonbold math in the Abstract.
% This preserves the distinction between vectors and scalars. However,
% if the conference you are submitting to favors bold math in the abstract,
% then you can use LaTeX's standard command \boldmath at the very start
% of the abstract to achieve this. Many IEEE journals/conferences frown on
% math in the abstract anyway.

% no keywords




% For peer review papers, you can put extra information on the cover
% page as needed:
% \ifCLASSOPTIONpeerreview
% \begin{center} \bfseries EDICS Category: 3-BBND \end{center}
% \fi
%
% For peerreview papers, this IEEEtran command inserts a page break and
% creates the second title. It will be ignored for other modes.
\IEEEpeerreviewmaketitle

\section{Introduction}

Air quality is a topic which is facing much publicity due to its prevalence in law and media, especially regarding pollution laws leading to driving bans. Poor air quality caused by industry and daily life leads to health rsisk andadversely impacts the environment. Inhabitants of big cities are particularly affectefd by air pollutants, which might manifest themselves in smog or an industrial smell.
Meanwhile, the concept of internet of things devices is on the rise, enabled by increased computation power, cost and energy efficient sensors and the availability of a fast cellular network.
\newline
Due to this, the city of Perlheim may want to consider to implement an air quality control system based on a multitude of sensors in a connected network. Thereby, aggregated and location-based information about the current air quality can be provided.

\section{Existing approaches}
It should be noted that measuring air quality as part of a smart city is no new approach, but has already been discussed in research \cite{Nagaraj} and has also been implemented in real-world projects. Before describing our idea for the city of Perheim, a brief overview of existing research and projects is given, futher showing that an air quality information system is a sensible idea. However, our concept is not directly based on one of these existing approaches.

There is also already a basic air monitoring system in place in the Rhein-Neckar area \cite{LandesanstaltfurUmweltBadenWurttemberg}.
However, the new smart city air quality concept for Perheim differs in several aspects from the already existing system:
\begin{itemize} 
\item The existing air monitoring consists of very few sensors per area. However, the approach propsed in this paper makes use of a higher number of small sensors to enable fine-granular monitoring.
\item A new interconnected approach also allows for futher integrating more sophisticated smart city concepts regarding automatic traffic diversion and health care in the future.
\item Currently, the measurements are not easy to consume. With live monitoring and an intuitive user interface, the visibility of the current status of the air quality will be greatly improved for citizens.
\end{itemize} 

\section{Use cases}
Monitoring the air quality and providing information directly might prove useful in differerent use cases. It is also important to note that air quality is a very relevant topic for the city of Perlheim. Perlheim's downtown area is heavily frequented by cars. Furthermore, one of the world's biggest chemical companies, whose emissions and increased risk of toxic accidents and fires might also adversely impact the air quality, is headquartered in the neighboring city. 
\newline \newline
Each of the following use cases can be enabled by an air quality information system. The required technology and sensors will be covered in section \ref{sec:TechnicalImplementation}.

\subsection{Managing health risks}
With an air quality information system, citizens can base their decision of when to leave the house or where to go for daily-life activities on the air quality. This might prove to be especially helpful for people with severe allergies or other adverse reactions to air pollutants like asthma. Also, health institutions can use the data and the home address of a patient to possibly correlate health issues with air pollution and thus providing more precise diagnosis. The air quality information system can also be used to give direct feedback in case of an acute risk by wildfires or leaks at chemical plants. In both cases it is usually recommended to stay inside. Negative health risks resulting from a high concentration of some air pollutants are summarized in a report carried out by the World Health Organization and are as follows \cite{WorldHealthOrganization.2013}:

\begin{description} 
\item[O\textsubscript{3}] Long- and short-term excessive Ozone exposure can lead to respitory diseases and restrictions in activity.
\item[NO\textsubscript{2}] Nitrogen dioxide exposure can lead to bronchitic symptoms in asthmatic children as well as an increase in all-cause mortality.
\item[PM\textsubscript{2.5}] Long-term effects of excessive exposure to particulate matter of 2.5 micrometers or less can be cardiovascular diseases, lung cancer and an increased all-cause mortality.
\item[PM\textsubscript{10}] Inhaling particulate matter of 10 micrometers or less may result in asthma symptoms in asthmatic children and chronic bronchitis.
\end{description} 

Apart from these major air pollutants, there is a wide variety of other air toxics which may cause cancer, respitory reactions or other diseases \cite{UnitedStatesEnvironmentalProtectionAgency}.

It is also possible to  build on the air quality information system by integrating the generated data into other technical devices. For example, windows could be automatically closed, air filters regulated and car navigation system could propose routes based on the air quality. However, this is out-of-scope for a first technical implementation in Perheim.

\subsection{Environmental damage}
Poor air quality not only impacts human health, but also the well-being of animals and plants. A change in air quality can deeply affect an eco-system. Certain types of insects might decline in population, making room for pest and invasive species. TODO

\subsection{Enabling law execution}
The impact of poor air quality on health and the environment has also been recognized by law makers. There are several laws governing the air pollution for industry and car traffic. The most important EU directive is 2008/50/EC \cite{EurpoeanUnion.2008}. This directive provides legal limit values for certain pollutants such as particulate matter, nitrogen dioxide and carbon monoxide. An air quality control system can help to monitor these limits, take action when needed and control their execution. Unusual or extreme measurements also hint on law breaches for example in form of illegaly operarted facilities that are subject to approval. In this case, the local municipality is allowed to take action in compliance with \S20 and \S62 of the Bundes-Immissionsschutzgesetz (BImSchG) \cite{BundesrepublikDeutschland.1974}. In a more advanced smart city approach, the fine-granular air quality information will be integrated automatic traffic control. Thereby, automatic speed limits and the diversion of cars from heavily to less frequented areas and streets help to satisfy required pollution limits.

\subsection{Psychological effect}
Having a direct and real-time exposure to air quality data might result in an increased ecological awareness of citizens. Thereby, a voluntary switch from e.g. coal energy to a renewable energy provider or from environmental-unfriendly cars to electric vehicles might be initiated. In the same context, Perheim receives an image boost, showing the city as being environmental friendly and digitally advanced.

\section{Economic feasability}
While the initial cost of sensors is not expected to be exceptionally high, maintenance and installation must also be accounted for. Several different models to covering the costs are possible.

\subsection{Tax-payer funded service}
In the most straight-forward approach, the local municipality and thus indirectly the tax payer is responsible for buying and installing the sensors. The service is then provided to all citizens of Perlheim free of charge. This is a sensible idea because not only do the citizens benefit from the air quality information system, but also the municipality itself.

\subsection{Subscription-based service}
Like other cloud and internet of things services, the air quality information could be provided as a subscription for a small fee of a few euros per month. However, this bears the disadvantage that most people might not be interested in the service or do not consider the price-value ratio as attractive. It could also be possible to keep some measured values free and enabling additional information based on more specific sensors for a small fee.
As discussed before, an air quality information system could be used as a foundation for other products such as smart home devices and health care software. Therefore, a seperate subscription for commercial use might be reasonable.

\subsection{Voluntary contributions}
The sensors could also be sold to the citizens at a discount. Thereby, Perheim's inhabitants can contribute voluntarily by buying the sensors and installing them outside their home. Through this model, a fine granular network can be achieved and the city only needs to cover the costs for the discount and servers. However, this approach entirely relies on a wide participation rate in the population.

\subsection{Data as a product}
Historical and live data might be interesting for companies and researches. Selling complete data sets might complement one of the other models. 

\section{Technical implementation}\label{sec:TechnicalImplementation}
The basic idea is, to build an air quality sensor out of an Arduino and multiple other sensors, distribute multiple of those air quality sensors throughout Perheim, gather the measured data into a central database and display this data into an easy to use and visually appealing application. The following chapters will cover each of these steps in detail.

\subsection{Building the air quality sensor}
First of all, it has to be determined what the final air quality sensor should measure. The European Union defined the European Air Quality Index (EAQI) in 2017 and it is similar to most of Air Quality Index from other governments around the world, like the U.S. Environmental Protection Agency. It provides a set of air quality index levels: good, fair, moderate, poor and very poor. Furthermore, it defines multiple components, such as Ozone, particulate matter, Sulfur dioxide, Carbon monoxide and Nitrous oxide, that are getting measured and it provides a table to calculate the EAQI with the measured data of these components. Since the EAQI was created by the EU it is used throughout Europe and will also be used as a basis for the air quality sensor in Perheim(Germany). As the goal is to distribute a lot of air quality sensors in Perheim, the first generation of air quality sensors will be built with cheaper Sensors to make them more affordable. In a later stage, when the developed system got accepted and gets used frequently, sensors can be more expensive and of higher quality standards. With the focus on cheaper products, the following sensors will be used for the first-generation air quality sensors:
\begin{itemize}
\item Ozone: SainSmart MQ131
\item Particulate matter: Grove Dust Sensor
\item Sulfur dioxide: Mq2 Gas Sensor
\item Carbon monoxide: Mq9 Gas Sensor
\item Nitrous oxide: MICS-2714
\end{itemize}
With these sensors all the components of the EAQI can be measured. Other important factors that play a crucial role to the severeness of the EAQI are the current temperature and humidity. That’s why a temperature and humidity sensor will also be integrated into the air quality sensor to improve the calculations. In the first iteration the SODIAL - Temperature and Relative Humidity Sensor will be used. Another point is, that the final application has to use the location of each sensor to display the measured data at the corresponding site. Therefore, each air quality sensor will be equipped with a NEO-6M GPS sensor to send the Gps coordinates of its location with each measured data set. At last the internet module FONA Mini Cell GSM Breakout SMA will be utilized to connect the Arduino to the internet and send the measured sensor data to a central system for further processing. All these hardware components will be soldered to a soldering board and thereby connected to an Arduino. The final construct will be placed in a weatherproof box that has three cutouts. A 5-Volt fan will be placed in the first cutout to blow the air through the box, along all the sensors and finally out of the second cutout. The third cutout is for a power supply for the Arduino. With that procedure many air quality sensors will be manufactured. The estimated cost of one sensor without the box is around 150. The needed Arduino coding to read all the measurements is not part of this document.

\subsection{Creating a backend to process incoming data and manage air quality sensors}
The backend system that provides an application programming interface (API) for the sent measurements of the sensors described previously will be covered in this section. This API can be implemented in many ways, e.g. as a Representational State Transfer (REST) API or as a create, read, update, and delete (CRUD) API, but the detailed implementation of this API won’t be discussed in this document. The API from the backend to receive the sent measurements of the sensors will provide the following functionalities:
\begin{itemize}
\item Receive a set of data containing the measurements, a timestamp and location from a sensor in JavaScript Object Notation (JSON) format
\item Save received data into a persistence as one tuple of the corresponding database table
\item Error handling in the event of wrong message formats
\item  Propagate crucial information if sensor measurements could be hazardous to health
\end{itemize}


\subsection{Creating a Frontend to display gathered data}
Subsubsection text here.

\section{Test Section}

Simple table example:
\begin{tabular}{ c | c | c }
	\hline			
	1 & 2 & 3 \\
	4 & 5 & 6 \\
	7 & 8 & 9 \\
	\hline  
\end{tabular}

% An example of a floating figure using the graphicx package.
% Note that \label must occur AFTER (or within) \caption.
% For figures, \caption should occur after the \includegraphics.
% Note that IEEEtran v1.7 and later has special internal code that
% is designed to preserve the operation of \label within \caption
% even when the captionsoff option is in effect. However, because
% of issues like this, it may be the safest practice to put all your
% \label just after \caption rather than within \caption{}.
%
% Reminder: the "draftcls" or "draftclsnofoot", not "draft", class
% option should be used if it is desired that the figures are to be
% displayed while in draft mode.
%
%\begin{figure}[!t]
%\centering
%\includegraphics[width=2.5in]{myfigure}
% where an .eps filename suffix will be assumed under latex, 
% and a .pdf suffix will be assumed for pdflatex; or what has been declared
% via \DeclareGraphicsExtensions.
%\caption{Simulation Results}
%\label{fig_sim}
%\end{figure}

% Note that IEEE typically puts floats only at the top, even when this
% results in a large percentage of a column being occupied by floats.


% An example of a double column floating figure using two subfigures.
% (The subfig.sty package must be loaded for this to work.)
% The subfigure \label commands are set within each subfloat command, the
% \label for the overall figure must come after \caption.
% \hfil must be used as a separator to get equal spacing.
% The subfigure.sty package works much the same way, except \subfigure is
% used instead of \subfloat.
%
%\begin{figure*}[!t]
%\centerline{\subfloat[Case I]\includegraphics[width=2.5in]{subfigcase1}%
%\label{fig_first_case}}
%\hfil
%\subfloat[Case II]{\includegraphics[width=2.5in]{subfigcase2}%
%\label{fig_second_case}}}
%\caption{Simulation results}
%\label{fig_sim}
%\end{figure*}
%
% Note that often IEEE papers with subfigures do not employ subfigure
% captions (using the optional argument to \subfloat), but instead will
% reference/describe all of them (a), (b), etc., within the main caption.


% An example of a floating table. Note that, for IEEE style tables, the 
% \caption command should come BEFORE the table. Table text will default to
% \footnotesize as IEEE normally uses this smaller font for tables.
% The \label must come after \caption as always.
%
%\begin{table}[!t]
%% increase table row spacing, adjust to taste
%\renewcommand{\arraystretch}{1.3}
% if using array.sty, it might be a good idea to tweak the value of
% \extrarowheight as needed to properly center the text within the cells
%\caption{An Example of a Table}
%\label{table_example}
%\centering
%% Some packages, such as MDW tools, offer better commands for making tables
%% than the plain LaTeX2e tabular which is used here.
%\begin{tabular}{|c||c|}
%\hline
%One & Two\\
%\hline
%Three & Four\\
%\hline
%\end{tabular}
%\end{table}


% Note that IEEE does not put floats in the very first column - or typically
% anywhere on the first page for that matter. Also, in-text middle ("here")
% positioning is not used. Most IEEE journals/conferences use top floats
% exclusively. Note that, LaTeX2e, unlike IEEE journals/conferences, places
% footnotes above bottom floats. This can be corrected via the \fnbelowfloat
% command of the stfloats package.



\section{Conclusion}
The conclusion goes here.




% conference papers do not normally have an appendix


% use section* for acknowledgement


% trigger a \newpage just before the given reference
% number - used to balance the columns on the last page
% adjust value as needed - may need to be readjusted if
% the document is modified later
%\IEEEtriggeratref{8}
% The "triggered" command can be changed if desired:
%\IEEEtriggercmd{\enlargethispage{-5in}}

% references section

% can use a bibliography generated by BibTeX as a .bbl file
% BibTeX documentation can be easily obtained at:
% http://www.ctan.org/tex-archive/biblio/bibtex/contrib/doc/
% The IEEEtran BibTeX style support page is at:
% http://www.michaelshell.org/tex/ieeetran/bibtex/
%\bibliographystyle{IEEEtran}
% argument is your BibTeX string definitions and bibliography database(s)
%\bibliography{IEEEabrv,../bib/paper}
%
% <OR> manually copy in the resultant .bbl file
% set second argument of \begin to the number of references
% (used to reserve space for the reference number labels box)
%\begin{thebibliography}{1}

%\bibitem{IEEEhowto:kopka}
%H.~Kopka and P.~W. Daly, \emph{A Guide to \LaTeX}, 3rd~ed.\hskip 1em plus
% 0.5em minus 0.4em\relax Harlow, England: Addison-Wesley, 1999.


%\end{thebibliography}

\bibliographystyle{IEEEtran}
\bibliography{literature}



% that's all folks
\end{document}


